%%%%%%%%%%%%%%%%%%%%%%%%%%%%%%%%%%%%%%%%%%%%%%%%%%%%%%%%%%%%%%%%%%%%%%%%%%%%%%%
%% Project:     Los Alamos Message Passing Interface (LA-MPI)
%% Descr:       A reliable, resilient message passing library
%% Author:      L. Dean Risinger jr.
%% $Id$
%%%%%%%%%%%%%%%%%%%%%%%%%%%%%%%%%%%%%%%%%%%%%%%%%%%%%%%%%%%%%%%%%%%%%%%%%%%%%%%

\documentclass[english]{article}
\usepackage[latin1]{inputenc}
\usepackage{babel}

%% do we have the `fancyhdr' package?
\IfFileExists{fancyhdr.sty}{
\usepackage[fancyhdr,pdf]{latex2man}
}{
%% do we have the `fancyheadings' package?
\IfFileExists{fancyheadings.sty}{
\usepackage[fancy,pdf]{latex2man}
}{
\usepackage[nofancy,pdf]{latex2man}
\message{no fancyhdr or fancyheadings package present, discard it}
}}

\setDate{2003/09/30}    %%%% must be manually set, if rcsinfo is not present
\message{package rcsinfo not present, discard it}

\setVersionWord{Version:}  %%% that's the default, no need to set it.
\setVersion{1.3.5}

\begin{document}

\begin{Name}{1}{mpirun}{LA-MPI Development team}{LA-MPI Reliable MPI}{Los Alamos Message Passing Library}

  \Prog{mpirun} - run MPI programs using the Los Alamos Message Passing library, LA-MPI.  
\end{Name}

\section{Synopsis}
%%%%%%%%%%%%%%%%%%

\Prog{mpirun} [\Opt{-n}\Bar\Opt{-np} \Arg{#,...}] \oOptArg{-N}{ #} \oOptArg{-H}{ hostlist} \oOptArg{-s}{ mpirunhost}
\oOptArg{-d}{ wdir} \oOptArg{-dapp}{ edir} \oOptArg{-dev}{ netpaths} \oOpt{-threads} \oOpt{-t} \oOpt{-crc} \oOptArg{-qf}{ flaglist}
\Arg{executable} [\Arg{args ...}]

\Prog{mpirun} -h\Bar-help

\Prog{mpirun} -version

\Prog{mpirun} -list-options

\section{Description}
%%%%%%%%%%%%%%%%%%%%%
\Prog{mpirun} starts the specified executables with LA-MPI as the MPI layer. LA-MPI is a network fault-tolerant message passing 
system designed for the challenges of the terascale cluster environment.

\section{Options}
%%%%%%%%%%%%%%%%%
\begin{Description}\setlength{\itemsep}{0cm}
\item[\Opt{-n}\Bar\Opt{-np} \Arg{ number of processes}] A single value specifies the total number of processes, while a 
comma-delimited list of numbers specifies the number of processes on each host or machine. 
\item[\OptArg{-N}{ number of hosts}] The number of hosts to use.
\item[\OptArg{-H}{ hostlist}] A comma-delimited list of name(s) of hosts to use.
\item[\OptArg{-s}{ mpirunhost}] A preferred IP interface name or address for TCP/IP administrative and UDP data traffic.
\item[\OptArg{-d}{ wdir}] A comma-delimited list of one or more working directories for spawned processes.
\item[\OptArg{-dapp}{ edir}] A comma-delimited list of one or more directories where the \Arg{executable} is located.
\item[\OptArg{-dev}{ netpaths}] A colon-delimited list of one or more network paths to use for this MPI job. Currently the choices
are some combination of "hippi800", "udp", and "quadrics" (depending upon your environment).
\item[\Opt{-threads}] Turn on thread safe code.
\item[\Opt{-t}] Turn on tagging of standard output and error lines with either GR[HR.LR] (GR = global rank, HR = host rank, LR = local rank), 
or simply HR (on RMS/prun systems).
\item[\Opt{-crc}] Use 32-bit CRCs instead of 32-bit additive checksums (the LA-MPI default) for application to application data integrity
where applicable.
\item[\OptArg{-qf}{ flaglist}] A comma-delimited list of keywords that affect operation on Quadrics networks. The keywords that are supported
are "noack", "ack", "nochecksum", and "checksum". "noack" disables retransmission of data, but allows data sends to complete upon local notification
by the Quadrics Elan adapters. "nochecksum" disables all LA-MPI CRC/checksum generation and verification; in this mode, data integrity guarantees are
provided only at the Elan adapter level. The defaults are "ack", and "checksum" for a LA-MPI library built with reliability mode enabled. This option
is likely to change in a future release.
\item[\Opt{-h}\Bar\Opt{-help}] Print a brief help message and exit.
\item[\Opt{-version}] Print version information and exit.
\item[\Opt{-list-options}] Print a longer help message with all options and exit.
%\item[\Opt{-d, --debug}] Run in debug mode.
%\item[\OptArg{--define}{ VAR=VAL}] Define an architecture specific
%  variable.
%\item[\OptArg{--env}{ VAR=VAL}] Define an environment variable to 
%  pass to the application.
%\item[\OptArg{--host}{ HOSTNAME:args}] Name of a host to use 
%  (where implemented) and optionally, the name of an executable to run on this host.
%\item[\OptArg{-N, --nhosts}{ NHOSTS}] Number of hosts to use.
%\item[\OptArg{-n, --nprocs}{ NPROCS}] Number of processes to start.
%\item[\Opt{-t, --tag}] Tag application output with process rank (and
%  host rank if appropriate).
%\item[\Opt{-v, --verbose}] Verbose output.
\end{Description}

There are many more options, but most are not of interest to the typical user. If you need more details on mpirun options,
please send email to \Email{lampi-support@lanl.gov}. 

\section{Arguments}
%%%%%%%%%%%%%%%%%%%
The \Arg{executable} must be either a single executable or a comma-delimited list of executables for each host or machine 
(which is not supported for all environments). Executable arguments apply to all specified executables. 

\section{Examples}
%%%%%%%%%%%%%%%%%%
\begin{enumerate}\setlength{\itemsep}{2cm}
\item[1. \textbf{mpirun dcalc -f foo}] On systems with LSF, this command can be used to
start \Prog{dcalc -f foo} on some number of hosts and processors preassigned by LSF.
\item[2. \textbf{mpirun -n 64 dcalc -f foo}] On systems with LSF, this command will start
64 processes of \Prog{dcalc -f foo} on either all or some subset of preassigned hosts and processors. If it is a subset,
then mpirun attempts to use all of the preassigned processors on a host before allocating from the
next reserved host's allocation. On systems with rsh spawning only, this command will start 64 processes on the
local host or machine.
\item[3. \textbf{mpirun -n 64 -N 40 dcalc -f foo}] On systems with standard LSF, this command will start
64 processes on 40 preassigned hosts as explained in example #2. On systems with LSF/RMS, mpirun will
start 64 processes on 40 hosts using \Cmd{prun}{1} in a block assignment mode. This command will not work
on systems with rsh spawning only, as there is no specific host information available.
\item[4. \textbf{mpirun -n 66 -H n1,n2,n3,n4 dcalc -f foo}] This command will start 66 processes on the 4 named hosts.
If the system supports LSF, then the assignment will be made using the processors available on each host. If the
system only support rsh spawning, then mpirun will assign as close as possible the same number of processes per host
(in this example, 17 processes will be started on n1 and n2, and 16 on n3 and n4).
\item[5. \textbf{mpirun -dev udp -t -np 2,10,30,22 -h n1,n2,n3,n4 dcalc -f foo}] On systems with rsh spawning, this command will 
start 2 processes on n1, 10 on n2, 30 on n3, and 22 on n4. Prefix tagging of standard output and
standard error will be done, and only UDP/IP will be used for intercommunication between hosts.
\end{enumerate}

\section{Files}
%%%%%%%%%%%%%%%
\begin{Description}\setlength{\itemsep}{0cm}
\item[\File{mpirun}] Job spawner using LA-MPI.
\item[\File{libmpi.so}] LA-MPI shared-object library.
\item[\File{mpi.h}] LA-MPI main C header file.
\item[\File{mpif.h}] LA-MPI main Fortran header file.
\end{Description}

\section{Diagnostics}
%%%%%%%%%%%%%%%%%%%%%
LA-MPI has an abundant number of error and diagnostic messages that will not
always be very clear to the end user. If your job dies unexpectedly, please
check to see if a file, \File{lampi.log}, has been generated in your working
directory.  All severe warnings and errors are logged to this file and standard
error.

As with any well-defined standard, it is very easy to compile with the wrong
mpi.h or mpif.h header files (especially on systems with more than one MPI
library). The result is unpredictable, but nearly always bad. Please be careful.

On systems that support job spawning via \Prog{rsh}, it is possible to have
difficulties if your \Prog{csh} startup environment (e.g., .cshrc problems)
reports errors for shells that are not login shells. \Prog{rsh} spawns a
\Prog{csh} to set certain environment variables, and start the first executable
on a host. Future versions of LA-MPI will allow different shells to be used.

\section{Reporting Bugs}
%%%%%%%%%%%%%%%%%%%%%%%%
Please send all bug reports to your friendly neighborhood developer at
\Email{lampi-support@lanl.gov}.

\section{Version}
%%%%%%%%%%%%%%%%%
Version: \Version\ of \Date.

\section{License}
%%%%%%%%%%%%%%%%%%%%%%%%%%%%%%%
\begin{description}

Copyright 2002-2003. The Regents of the University of California. This
material was produced under U.S. Government contract W-7405-ENG-36 for
Los Alamos National Laboratory, which is operated by the University of
California for the U.S. Department of Energy. The Government is
granted for itself and others acting on its behalf a paid-up,
nonexclusive, irrevocable worldwide license in this material to
reproduce, prepare derivative works, and perform publicly and display
publicly. Beginning five (5) years after October 10,2002 subject to
additional five-year worldwide renewals, the Government is granted for
itself and others acting on its behalf a paid-up, nonexclusive,
irrevocable worldwide license in this material to reproduce, prepare
derivative works, distribute copies to the public, perform publicly
and display publicly, and to permit others to do so. NEITHER THE
UNITED STATES NOR THE UNITED STATES DEPARTMENT OF ENERGY, NOR THE
UNIVERSITY OF CALIFORNIA, NOR ANY OF THEIR EMPLOYEES, MAKES ANY
WARRANTY, EXPRESS OR IMPLIED, OR ASSUMES ANY LEGAL LIABILITY OR
RESPONSIBILITY FOR THE ACCURACY, COMPLETENESS, OR USEFULNESS OF ANY
INFORMATION, APPARATUS, PRODUCT, OR PROCESS DISCLOSED, OR REPRESENTS
THAT ITS USE WOULD NOT INFRINGE PRIVATELY OWNED RIGHTS.

Additionally, this program is free software; you can distribute it
and/or modify it under the terms of the GNU Lesser General Public
License as published by the Free Software Foundation; either version 2
of the License, or any later version.  Accordingly, this program is
distributed in the hope that it will be useful, but WITHOUT ANY
WARRANTY; without even the implied warranty of MERCHANTABILITY or
FITNESS FOR A PARTICULAR PURPOSE.  See the GNU Lesser General Public
License for more details.

\end{description}

\section{Authors}
%%%%%%%%%%%%%%%%
\noindent
Rob T. Aulwes, David J. Daniel, Nehal N. Desai, Richard L. Graham, 
L. Dean Risinger, Mitchel W. Sukalski, Mark A. Taylor

\LatexManEnd

\end{document}
